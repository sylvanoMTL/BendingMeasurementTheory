\subsubsection{Thermal Amplification}
\label{sec:thermal_amplification}

Temperature variations do not only induce direct thermal expansion but also 
amplify existing imperfections in the measurement system. The resulting thermal 
effect can be represented as an effective change in both the measurement gain 
and offset.

The measured strain at temperature $T$ is expressed as:
\begin{equation}
\varepsilon_{\mathrm{meas}}(T)
=
A_T(T)\,\varepsilon_{\mathrm{mech}}(T_0)
+
\varepsilon_{\mathrm{offset}}(T)
\label{eq:thermal_amplification_general}
\end{equation}
where:
\begin{itemize}
  \item $\varepsilon_{\mathrm{mech}}(T_0)$ is the mechanical bending strain at the reference temperature $T_0$,
  \item $A_T(T)$ is a dimensionless thermal amplification factor,
  \item $\varepsilon_{\mathrm{offset}}(T)$ is a temperature-dependent offset
        caused by assembly pre-stress, twist amplification and differential expansion.
\end{itemize}

For small excursions around $T_0$, the amplification factor may be linearised as:
\begin{equation}
A_T(T)
\approx
1 + k_T\,(T - T_0)
\label{eq:thermal_gain_linear}
\end{equation}
where $k_T$ [K$^{-1}$] is an effective thermal amplification coefficient to be obtained from calibration.

The linear approximation in Eq.~\eqref{eq:thermal_gain_linear} assumes that the
operating temperature remains sufficiently close to the reference temperature
$T_0$ such that higher–order thermal effects are negligible. This corresponds
to a first–order Taylor expansion of $A_T(T)$, which is appropriate for modest
temperature excursions where:
\begin{equation}
|T - T_0| \ll \frac{1}{|k_T|}
\end{equation}
In the context of this system, the validity of the linear model will be
verified experimentally during temperature calibration. If significant
non-linear behaviour is observed, additional higher–order terms in temperature
may be introduced.




Substituting Eq.~\eqref{eq:thermal_gain_linear} into
Eq.~\eqref{eq:thermal_amplification_general} gives:
\begin{equation}
\varepsilon_{\mathrm{meas}}(T)
\approx
\left[1 + k_T\,(T - T_0)\right]\varepsilon_{\mathrm{mech}}(T_0)
+
\varepsilon_{\mathrm{offset}}(T)
\label{eq:thermal_amplification_final}
\end{equation}

Equation~\eqref{eq:thermal_amplification_final} shows that thermal effects act
as a multiplicative amplification of the mechanical strain measurement
in addition to an additive temperature-dependent offset.
Both parameters $k_T$ and $\varepsilon_{\mathrm{offset}}(T)$ 
must be determined experimentally during calibration.
