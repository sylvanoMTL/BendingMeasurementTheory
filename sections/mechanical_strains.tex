\section{Calculation of Mechanical Strains}
% ============================================
\label{sec:mechanical_strains}

This section presents detailed calculations of the mechanical bending strains $\varepsilon_{\text{mech}}(x,y)$ for the measurement beam under two different boundary condition assumptions. These calculations provide the theoretical basis for interpreting strain gauge measurements and understanding the relationship between applied forces and measured strains.

The mechanical strain component $\varepsilon_{\text{mech}}(T,F)$ defined in Eq.~\eqref{eq:strain_decomposition} is derived from classical beam theory applied to the specific geometry and loading conditions of the rowing oar system. Two theoretical models are considered to bracket the range of expected behavior:

\begin{itemize}
    \item \textbf{Theory 1 (Section~\ref{sec:theory1}):} Assumes a clamped boundary condition at the oarlock, representing a rigid constraint.
    \item \textbf{Theory 2 (Section~\ref{sec:theory2}):} Assumes pin support boundary conditions at both the oarlock and at the end of the outboard shaft, representing a more flexible support system.
\end{itemize}

Both theories use the same fundamental beam equations (Section~\ref{sec:beam_bending_theory}) but differ in their boundary conditions, leading to different deflection profiles $w(x)$ and consequently different strain distributions along the beam.

% Include subsections for Theory 1 and Theory 2
\subsection{Theory 1: Clamped Boundary Condition at Oarlock}
% ============================================
\label{sec:theory1}

\subsubsection{Boundary Conditions}

Theory 1 assumes the following boundary conditions:

\begin{itemize}
    \item \textbf{At the oarlock} ($x = 0$): The oar shaft is clamped, preventing both displacement and rotation:
    \begin{align}
    w_s(0) &= 0 \quad \text{(no vertical displacement)} \\
    \theta_s(0) &= 0 \quad \text{(no rotation, clamped)}
    \end{align}
    
    \item \textbf{At the handle} ($x = x_F$): A vertical force $F$ is applied, with no moment:
    \begin{align}
    V_s(x_F) &= -F \quad \text{(applied shear force)} \\
    M_s(x_F) &= 0 \quad \text{(free end, no moment)}
    \end{align}
\end{itemize}

See Figure~\ref{fig:theory1_boundary_conditions} for a visual representation of the boundary conditions.

\subsubsection{Deflection Solution}

\textbf{[TBD: Derivation of deflection $w_s(x)$ for clamped-free beam with point load]}

\subsubsection{Moment Distribution}

\textbf{[TBD: Calculation of bending moment $M_s(x)$ along the shaft]}

\subsubsection{Strain at Beam Location}

\textbf{[TBD: Calculation of mechanical strain $\varepsilon_{\text{mech}}$ at the beam gauge locations]}

\subsubsection{Numerical Example}

\textbf{[TBD: Numerical example with typical force values]}

% ============================================

\subsection{Theory 2: Pin Support Boundary Conditions}
% ============================================
\label{sec:theory2}

\subsubsection{Boundary Conditions}

Theory 2 assumes the following boundary conditions with pin supports at two locations:

\begin{itemize}
    \item \textbf{At the oarlock} ($x = 0$): Pin support prevents displacement but allows rotation:
    \begin{align}
    w_s(0) &= 0 \quad \text{(no transverse displacement)} \\
    M_s(0) &= 0 \quad \text{(no moment, pin allows rotation)}
    \end{align}
    
    \item \textbf{At the end of outboard shaft} ($x = -(L_{\text{out}} - L_{\text{blade}}) = -1570$ mm): Pin support prevents displacement but allows rotation:
    \begin{align}
    w_s(-(L_{\text{out}} - L_{\text{blade}})) &= 0 \quad \text{(no transverse displacement)} \\
    M_s(-(L_{\text{out}} - L_{\text{blade}})) &= 0 \quad \text{(no moment, pin allows rotation)}
    \end{align}
    
    \item \textbf{At the handle} ($x = x_F$): A transverse force $F$ is applied, with no moment:
    \begin{align}
    V_s(x_F) &= -F \quad \text{(applied shear force)} \\
    M_s(x_F) &= 0 \quad \text{(free end, no moment)}
    \end{align}
\end{itemize}

See Figure~\ref{fig:theory2_boundary_conditions} for a visual representation of the boundary conditions.

\subsubsection{Deflection Solution}

\textbf{[TBD: Derivation of deflection $w_s(x)$ for beam with two pin supports and point load]}

\subsubsection{Moment Distribution}

\textbf{[TBD: Calculation of bending moment $M_s(x)$ along the shaft]}

\subsubsection{Reaction Forces}

\textbf{[TBD: Calculation of reaction forces at the two pin support locations]}

\subsubsection{Strain at Beam Location}

\textbf{[TBD: Calculation of mechanical strain $\varepsilon_{\text{mech}}$ at the beam gauge locations]}

\subsubsection{Numerical Example}

\textbf{[TBD: Numerical example with typical force values]}

% ============================================


% ============================================
