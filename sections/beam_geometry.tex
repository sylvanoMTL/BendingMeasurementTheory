\section{Measurement Beam Geometry}
% ============================================

The measurement beam is a rectangular beam attached to the shaft by means of two rigid clamps at specified positions.

\subsection{Beam Dimensions and Position}

\begin{table}[ht]\label{tab:beam_dimensions}
\centering
\caption{Beam dimensions}
\begin{tabular}{llr}
\toprule
\textbf{Symbol} & \textbf{Description} & \textbf{Value} \\
\midrule
$L_b$ & Beam length (between clamp centers) & \SI{100}{mm} \\
$h_b$ & Beam height (bending direction, $y$-direction) & \SI{2}{mm} \\
$b$ & Beam width ($z$-direction) & \SI{12}{mm} \\
$x_b$ & Beam root position (first clamp) & \SI{200}{mm} \\
$e_b$ & Beam neutral axis eccentricity from shaft centerline & \SI{20}{mm} \\
$y_b$ & Beam neutral axis $y$-coordinate & $D_{o,s}/2 + e_b$ \\
\bottomrule
\end{tabular}
\end{table}

\textbf{Beam neutral axis position:} The beam neutral axis is located at $y_b$ measured from the global coordinate origin (shaft centerline at $y = 0$). This position is calculated as:
\begin{equation}
y_b = \frac{D_{o,s}}{2} + e_b
\label{eq:beam_neutral_axis}
\end{equation}

where $D_{o,s}/2$ is the shaft outer radius and $e_b$ is the distance from the shaft outer 
surface to the beam neutral axis. The beam extends from the shaft surface outward (in the $+y$ direction) 
to provide clearance for mounting strain gauges on both top and bottom surfaces.

\subsection{Beam Clamp Positions}

The beam is attached to the shaft at two locations:
\begin{align}
x_{b,1} &= x_b = \SI{200}{mm} \quad \text{(root clamp)} \\
x_{b,2} &= x_b + L_b = \SI{300}{mm} \quad \text{(tip clamp)}
\end{align}

\subsection{Clamp Rigidity and Assembly Tolerances}

The clamps are assumed to be \textbf{rigid connections} that prevent relative motion between the beam and shaft at the attachment points. However, two manufacturing and assembly phenomena must be considered:

\subsubsection{System Misalignment}

The entire beam assembly may be rotated around the shaft ($x$-axis) during installation. This misalignment angle, denoted $\phi_{\text{mis}}$, is expected to be within $\pm 1^\circ$. This creates a \textbf{static angular offset} of the entire measurement system, which may result in:

\begin{itemize}
    \item Apparent strain offset in gauge readings
    \item Coupling between bending and the misaligned measurement axes
\end{itemize}

\textbf{[TBD: Quantification of offset effect on strain measurements]}

\subsubsection{Initial Beam Twist}

The two clamps may not be perfectly aligned with each other during assembly, creating an initial twist in the beam. This twist angle, denoted $\phi_0$, represents the relative rotation between the root clamp (at $x = x_b$) and the tip clamp (at $x = x_b + L_b$) around the $x$-axis.

\textbf{Thermal amplification:} The differential thermal expansion between the aluminum beam and carbon shaft can amplify this initial twist. Over the operating temperature range ($\Delta T = 85$ K), the mismatch in thermal expansion creates additional torsional strain that compounds with $\phi_0$.

\textbf{Simplification:} The analysis assumes that twist-induced strains in the strain gauges are \textbf{negligible} compared to bending strains. This assumption should be verified for the expected values of $\phi_{\text{mis}}$ and $\phi_0$.

% ============================================
