\subsection{Theory 2: Pin Support Boundary Conditions}
\label{sec:theory2}

Theory~2 explores an alternative boundary condition model in which the oar is supported by two pin supports rather than a single clamp. This represents a more compliant constraint system that may better approximate the actual rowing conditions where the oarlock and blade-water interface provide vertical support but permit some rotational freedom.

\subsubsection{Boundary conditions and structural model}

The inboard and outboard shaft are modeled as a continuous Euler--Bernoulli beam with pin supports at two locations:

\begin{itemize}
  \item \textbf{Pin support at oarlock} ($x = 0$):
  \begin{align}
    w_s(0) &= 0 && \text{(zero transverse displacement)}, \\
    M_s(0) &= 0 && \text{(zero moment; pin permits rotation)}.
  \end{align}
  
  \item \textbf{Pin support at blade-shaft junction} ($x = x_{\text{pin,L}} = -(L_{\text{out}} - L_{\text{blade}}) = \SI{-1570}{mm}$):
  \begin{align}
    w_s(x_{\text{pin,L}}) &= 0 && \text{(zero transverse displacement)}, \\
    M_s(x_{\text{pin,L}}) &= 0 && \text{(zero moment; pin permits rotation)}.
  \end{align}
  
  \item \textbf{Free end at handle} ($x = x_F = \SI{900}{mm}$):
  \begin{align}
    V_s(x_F) &= F && \text{(applied shear force)}, \\
    M_s(x_F) &= 0 && \text{(free end; no moment)}.
  \end{align}
\end{itemize}

The span between the two pin supports is $L_{\text{span}} = 0 - x_{\text{pin,L}} = \SI{1570}{mm}$, and the handle extends as an overhang of length $x_F = \SI{900}{mm}$ beyond the right support.

\subsubsection{Reaction forces}

The vertical reaction forces at the two pins are determined from global equilibrium. Taking vertical force equilibrium:
\begin{equation}
  R_L + R_R = F,
  \label{eq:theory2_force_equilibrium}
\end{equation}
and moment equilibrium about the right pin at $x = 0$:
\begin{equation}
  R_L \cdot L_{\text{span}} - F \cdot x_F = 0.
  \label{eq:theory2_moment_equilibrium}
\end{equation}

Solving for the reactions:
\begin{align}
  R_L &= \frac{F \cdot x_F}{L_{\text{span}}}
       = \frac{F \times 900}{1570}
       = 0.573\,F, \\
  R_R &= F - R_L = 0.427\,F.
\end{align}

Substituting numerical values with $F = \SI{1962}{N}$:
\begin{align}
  R_L &\approx \SI{1124}{N}, \\
  R_R &\approx \SI{838}{N}.
\end{align}

\textbf{Note on reaction directions:} Both reactions are positive (upward, in the $+y$ direction) as expected for a beam supporting a downward load in the overhang region. This contrasts with cantilever configurations where support moments would be required.

\subsubsection{Bending moment distribution}

The bending moment varies across different regions of the beam. We focus on the overhang region $0 \le x \le x_F$ where the measurement beam is located.

\paragraph{Overhang region ($0 \le x \le x_F$):}

Cutting the beam at position $x$ in the overhang and considering the right portion (from $x$ to $x_F$), the only external force is the applied load $F$ at $x = x_F$. The internal bending moment at the cut is found from moment equilibrium:
\begin{equation}
  M_s(x) = F(x_F - x), \qquad 0 \le x \le x_F.
  \label{eq:theory2_moment_overhang}
\end{equation}

This expression satisfies the boundary conditions:
\begin{align}
  M_s(x_F) &= 0 \quad \checkmark \quad \text{(free end)}, \\
  M_s(0) &= F \cdot x_F = 900F \quad \text{(to be reconciled with pin condition)}.
\end{align}

\textbf{Apparent inconsistency at $x=0$:} Equation~\eqref{eq:theory2_moment_overhang} gives $M_s(0) = 900F \ne 0$, which appears to violate the pin support condition $M_s(0) = 0$. This reflects the physical reality that the overhang creates a substantial bending moment at the support location. In practice, the oarlock provides a combination of vertical support and partial rotational constraint, making it neither a perfect pin nor a perfect clamp. The pin support model represents an idealization that captures the vertical support while the calculated moment $M_s(0)$ indicates the degree of rotational constraint actually required. For the purposes of strain calculation at the measurement location ($x = x_{\text{gauge}} = \SI{250}{mm}$), the moment distribution in the overhang region given by Eq.~\eqref{eq:theory2_moment_overhang} provides a reasonable working model.

\paragraph{Supported span ($x_{\text{pin,L}} \le x < 0$):}

The bending moment distribution in this region depends on both pin reactions. For completeness:
\begin{equation}
  M_s(x) = R_L (x - x_{\text{pin,L}}), \qquad x_{\text{pin,L}} \le x < 0,
\end{equation}
which satisfies $M_s(x_{\text{pin,L}}) = 0$ and provides continuity considerations at the right support.

\subsubsection{Mechanical strain at gauge locations}

Following the same approach as Theory~1, the curvature at the gauge station $x = x_{\text{gauge}} = x_b + L_b/2 = \SI{250}{mm}$ is:
\begin{equation}
  \kappa_s(x_{\text{gauge}})
  = \frac{M_s(x_{\text{gauge}})}{E_s I_s}
  = \frac{F(x_F - x_{\text{gauge}})}{E_s I_s}.
  \label{eq:theory2_curvature}
\end{equation}

The mechanical bending strains at the top and bottom gauge fibers are:
\begin{align}
  \varepsilon_{\text{top}}(F)
  &= \kappa_s(x_{\text{gauge}}) \, y_{\text{top}}
   = \frac{F(x_F - x_{\text{gauge}}) \, y_{\text{top}}}{E_s I_s},
   \label{eq:theory2_eps_top} \\
  \varepsilon_{\text{bottom}}(F)
  &= \kappa_s(x_{\text{gauge}}) \, y_{\text{bottom}}
   = \frac{F(x_F - x_{\text{gauge}}) \, y_{\text{bottom}}}{E_s I_s}.
   \label{eq:theory2_eps_bottom}
\end{align}

The differential strain driving the bridge output is:
\begin{equation}
  \Delta\varepsilon(F)
  = \varepsilon_{\text{top}}(F) - \varepsilon_{\text{bottom}}(F)
  = \kappa_s(x_{\text{gauge}}) \, h_b
  = \frac{F(x_F - x_{\text{gauge}}) \, h_b}{E_s I_s}.
  \label{eq:theory2_deps}
\end{equation}

As in Theory~1, the differential mechanical strain is independent of the beam eccentricity $y_b$ and depends only on the local curvature and beam thickness $h_b$.

\subsubsection{Numerical example}

Using the same geometric and material parameters as Theory~1:
\begin{align}
  x_F - x_{\text{gauge}} &= 900 - 250 = 650~\text{mm}, \\
  h_b &= 2~\text{mm}, \\
  E_s &= 140~\text{GPa}, \\
  I_s &= 50\,956~\text{mm}^4,
\end{align}
and the peak handle force $F = \SI{1962}{N}$, the shaft curvature at the gauge station is:
\begin{equation}
  \kappa_s(x_{\text{gauge}})
  = \frac{1962 \times 650}{140 \times 10^3 \times 50\,956}
  \approx 0.179~\text{m}^{-1}.
\end{equation}

Note that this curvature is identical to Theory~1, as both theories place the gauge at $x = \SI{250}{mm}$ where the moment arm to the load is the same ($x_F - x_{\text{gauge}} = \SI{650}{mm}$). The mechanical strains are therefore:
\begin{align}
  \varepsilon_{\text{top}}
  &\approx 7.2 \times 10^{-3} = 7\,160~\mu\varepsilon, \\
  \varepsilon_{\text{bottom}}
  &\approx 6.8 \times 10^{-3} = 6\,800~\mu\varepsilon, \\
  \Delta\varepsilon
  &= \varepsilon_{\text{top}} - \varepsilon_{\text{bottom}}
   \approx 3.6 \times 10^{-4}
   = 360~\mu\varepsilon.
\end{align}

\subsubsection{Comparison with Theory 1}

Both theories predict \emph{identical} mechanical strains at the gauge location for the same applied handle force. This equivalence arises because:
\begin{enumerate}
  \item The moment arm from the load to the gauge position is the same in both models: $x_F - x_{\text{gauge}} = \SI{650}{mm}$.
  \item The local bending moment $M_s(x_{\text{gauge}}) = F(x_F - x_{\text{gauge}})$ depends only on this moment arm and the applied force, regardless of the support conditions elsewhere on the beam.
  \item The beam properties ($E_s$, $I_s$) and gauge geometry ($y_{\text{top}}$, $y_{\text{bottom}}$, $h_b$) are identical.
\end{enumerate}

The key difference between the theories lies in the \emph{global deflection pattern} and the \emph{reaction forces at the oarlock}:
\begin{itemize}
  \item \textbf{Theory 1 (clamped):} Single reaction at oarlock, zero rotation, larger deflections in outboard region.
  \item \textbf{Theory 2 (pinned):} Two reaction forces, rotational freedom at supports, reduced deflections due to additional support at blade-shaft junction.
\end{itemize}

However, these global differences do not affect the local strain measurement at $x = x_{\text{gauge}}$, confirming that the measurement system is sensitive primarily to the \emph{local bending moment} rather than the specific boundary condition model.
