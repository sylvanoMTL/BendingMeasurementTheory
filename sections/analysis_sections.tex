\section{Analysis and Calibration Results}
% ============================================
\label{sec:analysis}

This section presents the calibration results obtained from mechanical loading and thermal characterization experiments. The goal is to determine the coefficients that relate the measured strain $\varepsilon_{\text{meas}}(T,F)$ to the mechanical strain $\varepsilon_{\text{mech}}(T,F)$ while accounting for thermal and offset contributions as defined in Eq.~\eqref{eq:strain_decomposition}.

\subsection{Mechanical Calibration}
\label{sec:analysis_mechanical}

During mechanical calibration, the oar was held at approximately constant temperature $T \approx T_0$, such that:
\begin{equation}
\varepsilon_{\text{th}}(T_0) + \varepsilon_{\text{off}}(T_0) \approx 0
\end{equation}

Thus, Eq.~\eqref{eq:thermal_amplification_final} simplifies to:
\begin{equation}
\varepsilon_{\text{meas}}(T_0,F)
\approx
\varepsilon_{\text{mech}}(T_0,F)
\label{eq:mechanical_calibration_relation}
\end{equation}

The calibration curve $\varepsilon_{\text{meas}}$ vs.\ applied force $F$ exhibits a strong linear correlation. A first-order model was fitted:
\begin{equation}
\varepsilon_{\text{meas}}(T_0,F) = C_F \cdot F
\end{equation}
with calibration factor $C_F$ [\si{\microstrain\per\newton}] representing the mechanical sensitivity at $T_0$.

This establishes the baseline mechanical response contribution to Eq.~\eqref{eq:thermal_amplification_final}.

\subsection{Thermal Characterization}
\label{sec:analysis_thermal}

With the oar unloaded ($F = 0$), Eq.~\eqref{eq:thermal_amplification_final} becomes:
\begin{equation}
\varepsilon_{\text{meas}}(T,0)
=
\varepsilon_{\text{th}}(T)
+
\varepsilon_{\text{off}}(T)
\label{eq:thermal_characterization_equation}
\end{equation}

The measured output over temperature therefore reveals a combination of thermal expansion effects (predictable) and assembly/electronics offsets (less predictable). A linear model was fitted to $\varepsilon_{\text{meas}}(T,0)$ to estimate the dominant temperature-dependent trend:
\begin{equation}
\varepsilon_{\text{meas}}(T,0)
\approx
k_0
+
k_{\text{off}}(T - T_0)
\end{equation}
where:
\begin{itemize}
    \item $k_0$ is the residual strain at the reference temperature $T_0$,
    \item $k_{\text{off}}$ captures the dominant linear temperature dependence of the offset.
\end{itemize}

These coefficients contribute to modeling $\varepsilon_{\text{off}}(T)$ in Eq.~\eqref{eq:thermal_amplification_final}.

\subsection{Joint Temperature and Load Influence}
\label{sec:analysis_joint}

When both temperature changes and mechanical loads occur simultaneously, the measured strain follows:
\begin{equation}
\varepsilon_{\text{meas}}(T,F)
\approx
\left[1 + k_T\,(T - T_0)\right] C_F\, F
+
\varepsilon_{\text{th}}(T)
+
\varepsilon_{\text{off}}(T)
\end{equation}

The parameter $k_T$ was extracted from controlled tests where load was applied at different temperatures. A linear regression on the gain variation yielded:
\[
A_T(T) \approx 1 + k_T (T - T_0)
\]
allowing the separation of:
\begin{itemize}
    \item mechanical gain variation with temperature (via $k_T$)
    \item additive thermal and offset effects (via $\varepsilon_{\text{th}}$ and $\varepsilon_{\text{off}}$)
\end{itemize}

\subsection{Discussion}

The results demonstrate that:
\begin{itemize}
    \item The \textbf{mechanical calibration} is linear and stable near $T_0$.
    \item The \textbf{thermal characterization} reveals a temperature-dependent offset that must be compensated.
    \item A \textbf{joint thermo-mechanical model} is required to ensure accuracy across realistic operating temperatures.
\end{itemize}

To enable reliable performance during on-water operation, the coefficients $C_F$, $k_T$ and the characterization of $\varepsilon_{\text{off}}(T)$ must be incorporated into a real-time compensation algorithm.
