\section{Strain Gauge Specifications}
% ============================================

\subsection{Gauge Type and Configuration}

The measurement system uses four linear strain gauges arranged in a full Wheatstone bridge configuration.

\begin{table}[ht]
\centering
\caption{Strain gauge specifications}
\begin{tabular}{llr}
\toprule
\textbf{Parameter} & \textbf{Symbol} & \textbf{Value} \\
\midrule
Nominal resistance & $R_g$ & \SI{1000}{\ohm} $\pm$ \SI{3}{\ohm} \\
Gauge factor & $GF$ & $2.15$ \\
Maximum strain & -- & \SI{2000}{\microstrain} \\
Temperature coefficient of resistance & -- & \SI{+20}{ppm/K} \\
Backing material & -- & Polyimide \\
\bottomrule
\end{tabular}
\end{table}

All four strain gauges are located at the beam midpoint along the $x$-axis:
\begin{equation}
x_{\text{gauge}} = x_b + \frac{L_b}{2}
\label{eq:gauge_position}
\end{equation}

\subsection{Bridge Configuration}

During the drive phase, the rower applies a horizontal force at the handle in the $-y$ direction. This creates a bending moment that makes the measurement beam \emph{convex toward $+y$}. In this condition, the top surface is in tension and the bottom surface is in compression.

The four strain gauges form a full Wheatstone bridge configured for differential bending measurement:
\begin{itemize}
    \item \textbf{$R_1$ \& $R_3$ (Top surface):} located at $y = y_{\text{top}}$, $x = x_{\text{gauge}}$
    \begin{itemize}
        \item Experience \textbf{positive strain} (tension) during the drive phase.
    \end{itemize}
    \item \textbf{$R_2$ \& $R_4$ (Bottom surface):} located at $y = y_{\text{bottom}}$, $x = x_{\text{gauge}}$
    \begin{itemize}
        \item Experience \textbf{negative strain} (compression) during the drive phase.
    \end{itemize}
\end{itemize}

To ensure sensitivity to bending while rejecting common-mode strain (e.g., uniform axial or thermal strain),
each half-bridge contains one top and one bottom gauge:
\begin{itemize}
    \item Left voltage divider: $R_1$ (top) over $R_2$ (bottom)
    \item Right voltage divider: $R_3$ (bottom) over $R_4$ (top)
\end{itemize}

The bridge output voltage is:
\begin{equation}
V_{\text{out}} = V_{\text{ex}} \left(
    \frac{R_2}{R_1 + R_2}
    - \frac{R_4}{R_3 + R_4}
\right)
\label{eq:bridge_basic}
\end{equation}

with strain-dependent resistances:
\begin{align}
R_1 &= R_g \bigl(1 + GF \cdot \varepsilon_{\text{top}}\bigr) \\
R_2 &= R_g \bigl(1 + GF \cdot \varepsilon_{\text{bottom}}\bigr) \\
R_3 &= R_g \bigl(1 + GF \cdot \varepsilon_{\text{bottom}}\bigr) \\
R_4 &= R_g \bigl(1 + GF \cdot \varepsilon_{\text{top}}\bigr)
\end{align}

For small strains ($GF \cdot \varepsilon \ll 1$), this simplifies to the familiar linearized form:
\begin{equation}
\frac{V_{\text{out}}}{V_{\text{ex}}}
=
\frac{GF}{2}
\left(\varepsilon_{\text{top}} - \varepsilon_{\text{bottom}}\right)
\label{eq:bridge_output}
\end{equation}

where:
\begin{itemize}
    \item $V_{\text{out}}$ = bridge output voltage
    \item $V_{\text{ex}}$ = bridge excitation voltage (here \SI{3.3}{V})
    \item $\varepsilon_{\text{top}}$ = average strain on top surface (gauges $R_1$ \& $R_3$)
    \item $\varepsilon_{\text{bottom}}$ = average strain on bottom surface (gauges $R_2$ \& $R_4$)
\end{itemize}

\textbf{Drive-phase sign convention:}
\begin{itemize}
    \item $\varepsilon_{\text{top}} > 0$ (tension)
    \item $\varepsilon_{\text{bottom}} < 0$ (compression)
    \item $V_{\text{out}} > 0$ indicates bending consistent with the drive-phase loading.
\end{itemize}

Unless otherwise stated, $\varepsilon_{\text{top}}$ and $\varepsilon_{\text{bottom}}$ are understood to be the \emph{total} axial strains at the corresponding gauge locations. In terms of the decomposition introduced in Eq.~\eqref{eq:strain_decomposition}, they correspond to $\varepsilon_{\text{total}}(T,F)$ evaluated at $y = y_{\text{top}}$ and $y = y_{\text{bottom}}$, respectively.
Positive bridge output therefore corresponds directly to positive drive-phase force.


\subsection{Expected Strain Range}

Based on preliminary beam theory estimates (see Section~\ref{sec:beam_bending_theory}), the expected mechanical strain range at the gauge locations is on the order of:
\begin{itemize}
    \item $\varepsilon_{\text{mech}} \in [\SI{-500}{\microstrain}, \SI{+500}{\microstrain}]$ for typical rowing forces during the drive phase.
\end{itemize}

Thermal and offset contributions $\varepsilon_{\text{th}}(T)$ and $\varepsilon_{\text{off}}(T)$ may add bias and apparent gain variations, as discussed in Section~\ref{sec:thermal_amplification}.
