\subsection{Theory 1: Clamped Oarlock and Local Surface Strain}
\label{sec:theory1}

Theory~1 models the oar shaft as a uniform Euler--Bernoulli beam of length $x_F$
(clamped at the oarlock and free at the handle) with a transverse point load $F$
applied at $x = x_F$ during the drive phase. The aluminum measurement beam is
treated purely as a kinematic attachment that places the strain gauges at a
known distance from the shaft neutral axis. All strains derived in this section
are therefore \emph{purely mechanical bending strains} arising from the global
bending of the shaft.

Thermal and assembly contributions are handled separately in
Section~\ref{sec:theoretical_framework}.

\subsubsection{Shaft model and boundary conditions}

We consider the inboard shaft segment $0 \le x \le x_F$ with the following
boundary conditions at the oarlock and handle:
\begin{itemize}
  \item Clamped at the oarlock ($x = 0$):
  \begin{align}
    w_s(0) &= 0 && \text{(zero transverse displacement)}, \\
    \theta_s(0) = \left.\frac{\mathrm{d}w_s}{\mathrm{d}x}\right|_{x=0} &= 0
      && \text{(zero rotation)}.
  \end{align}
  \item Free at the handle ($x = x_F$), with a transverse force of magnitude $F$:
  \begin{align}
    V_s(x_F) &= F && \text{(shear force)}, \\
    M_s(x_F) &= 0 && \text{(zero bending moment)}.
  \end{align}
\end{itemize}

The shaft has bending stiffness $E_s I_s$, with $E_s$ and $I_s$ as defined in
Table~\ref{tab:nomenclature_material} and Eq.~(16).

\subsubsection{Bending moment, curvature and deflection of the shaft}

For a prismatic cantilever of length $x_F$ with a tip load $F$, the internal
bending moment distribution is
\begin{equation}
  M_s(x) = F\,(x_F - x), \qquad 0 \le x \le x_F.
  \label{eq:theory1_Mx}
\end{equation}
The maximum moment occurs at the clamped oarlock:
\begin{equation}
  M_s(0) = F x_F.
\end{equation}

The curvature of the shaft in the $x$--$y$ bending plane follows directly from
the Euler--Bernoulli relation
\begin{equation}
  \kappa_s(x) = \frac{\mathrm{d}^2 w_s}{\mathrm{d}x^2}
              = \frac{M_s(x)}{E_s I_s}
              = \frac{F\,(x_F - x)}{E_s I_s}.
  \label{eq:theory1_curvature}
\end{equation}

Integrating twice with the clamped boundary conditions at $x = 0$ yields the
standard cantilever expressions
\begin{align}
  \theta_s(x) &= \frac{\mathrm{d}w_s}{\mathrm{d}x}
  = -\frac{F}{2 E_s I_s}\,\bigl(2 x_F x - x^2\bigr), \\
  w_s(x) &=
  -\frac{F}{6 E_s I_s}\,x^2\,\bigl(3 x_F - x\bigr).
\end{align}
The maximum deflection at the handle is therefore
\begin{equation}
  w_s(x_F) = -\frac{F x_F^3}{3 E_s I_s}.
\end{equation}

\subsubsection{Mechanical strain at an arbitrary shaft fibre}

For small strains, the mechanical bending strain at a material fibre located a
distance $y$ from the shaft neutral axis (positive in the $+y$ direction) is
\begin{equation}
  \varepsilon_{\text{mech},s}(x,y)
  = \kappa_s(x)\,y
  = \frac{M_s(x)\,y}{E_s I_s}
  = \frac{F\,(x_F - x)\,y}{E_s I_s}.
  \label{eq:theory1_eps_generic}
\end{equation}
This expression is the core result: once the gauge position $(x,y)$ is known
relative to the shaft centreline, the local mechanical strain due to an applied
handle force $F$ follows directly.

\subsubsection{Gauge locations and effective radius}

The aluminum measurement beam is attached to the shaft by two clamps located at
\begin{align}
  x_{b,1} &= x_b, \\
  x_{b,2} &= x_b + L_b,
\end{align}
as defined in Table~\ref{tab:beam_dimensions}.
The four strain gauges are located at midspan of the beam,
\begin{equation}
  x_\text{gauge} = x_b + \frac{L_b}{2},
\end{equation}
so the relevant shaft curvature is $\kappa_s(x_\text{gauge})$.

In the radial ($y$) direction, the beam neutral axis lies at
\begin{equation}
  y_b = \frac{D_{o,s}}{2} + e_b,
\end{equation}
measured from the shaft centreline (see Eq.~(1)). The top and bottom gauge
fibres are located symmetrically about this neutral axis at
\begin{align}
  y_\text{top}    &= y_b + \frac{h_b}{2}, \\
  y_\text{bottom} &= y_b - \frac{h_b}{2}.
\end{align}

\subsubsection{Mechanical strain at the gauge locations}

Evaluating Eq.~\eqref{eq:theory1_eps_generic} at the gauge station
$x = x_\text{gauge}$ gives
\begin{equation}
  \kappa_s(x_\text{gauge})
  = \frac{F\,(x_F - x_\text{gauge})}{E_s I_s}.
\end{equation}
The purely mechanical bending strains at the top and bottom gauge fibres are
then
\begin{align}
  \varepsilon_\text{top}(F)
  &= \kappa_s(x_\text{gauge})\,y_\text{top}
   = \frac{F\,(x_F - x_\text{gauge})\,y_\text{top}}{E_s I_s},
   \label{eq:theory1_eps_top} \\
  \varepsilon_\text{bottom}(F)
  &= \kappa_s(x_\text{gauge})\,y_\text{bottom}
   = \frac{F\,(x_F - x_\text{gauge})\,y_\text{bottom}}{E_s I_s}.
   \label{eq:theory1_eps_bottom}
\end{align}

The bridge differential strain component is the difference between these two:
\begin{equation}
  \Delta\varepsilon(F)
  = \varepsilon_\text{top}(F) - \varepsilon_\text{bottom}(F)
  = \kappa_s(x_\text{gauge})\,(y_\text{top} - y_\text{bottom})
  = \kappa_s(x_\text{gauge})\,h_b.
  \label{eq:theory1_deps}
\end{equation}
Notably, the \emph{differential} mechanical strain driving the full bridge
output is \emph{independent} of the beam eccentricity $y_b$; it depends only on
the local curvature and beam thickness $h_b$.

In contrast, the \emph{absolute} (common-mode) strains at the gauge locations
scale with the eccentricity $y_b$:
\begin{equation}
  \varepsilon_\text{cm}(F)
  \approx \kappa_s(x_\text{gauge})\,y_b,
\end{equation}
which must remain within the strain gauge specification.

\subsubsection{Bridge response (mechanical contribution)}

Using the full-bridge configuration defined in Section~\ref{sec:bridge_configuration},
and for small strains, the normalized bridge output is
\begin{equation}
  \frac{V_\text{out}}{V_\text{ex}}
  = \frac{GF}{2}\,\bigl(\varepsilon_\text{top} - \varepsilon_\text{bottom}\bigr)
  = \frac{GF}{2}\,\Delta\varepsilon(F),
\end{equation}
with $\Delta\varepsilon(F)$ given by Eq.~\eqref{eq:theory1_deps}. Combining
Eqs.~\eqref{eq:theory1_curvature} and~\eqref{eq:theory1_deps}, the purely
mechanical bridge strain is therefore
\begin{equation}
  \Delta\varepsilon(F)
  = \frac{F\,(x_F - x_\text{gauge})\,h_b}{E_s I_s}.
  \label{eq:theory1_deps_final}
\end{equation}

\subsubsection{Numerical example}

Using the geometric and material parameters in
Tables~\ref{tab:shaft_dimensions}--\ref{tab:beam_dimensions}:
\begin{align}
  x_F &= 900~\text{mm}, &
  x_b &= 200~\text{mm}, &
  L_b &= 100~\text{mm}, \\
  h_b &= 2~\text{mm}, &
  D_{o,s} &= 38~\text{mm}, &
  e_b &= 20~\text{mm}, \\
  E_s &= 140~\text{GPa}, &
  I_s &= 50\,956~\text{mm}^4,
\end{align}
and the peak handle force $F = 1962~\text{N}$, we have
\begin{align}
  x_\text{gauge} &= x_b + \frac{L_b}{2} = 250~\text{mm}, \\
  y_b &= \frac{D_{o,s}}{2} + e_b = 39~\text{mm}, \\
  y_\text{top} &= 40~\text{mm}, \qquad
  y_\text{bottom} = 38~\text{mm}.
\end{align}
The shaft curvature at the gauge station is
\begin{equation}
  \kappa_s(x_\text{gauge})
  = \frac{F\,(x_F - x_\text{gauge})}{E_s I_s}
  \approx 0.179~\text{m}^{-1},
\end{equation}
yielding the mechanical strains
\begin{align}
  \varepsilon_\text{top}
  &\approx 7.2 \times 10^{-3} = 7\,160~\mu\varepsilon, \\
  \varepsilon_\text{bottom}
  &\approx 6.8 \times 10^{-3} = 6\,800~\mu\varepsilon, \\
  \Delta\varepsilon
  &= \varepsilon_\text{top} - \varepsilon_\text{bottom}
   \approx 3.6 \times 10^{-4}
   = 360~\mu\varepsilon.
\end{align}

In this formulation, the large absolute strains at the gauge locations are
recognized as \emph{true mechanical bending strains} at the eccentric radius
$y_b$, while the full-bridge output is driven by the differential component
$\Delta\varepsilon(F)$ given by Eq.~\eqref{eq:theory1_deps_final}. Comparison
of these theoretical values with the calibrated sensitivity in
Section~\ref{sec:mechanical_calibration} provides a direct check on the
effective gauge radius and clamp stiffness.
