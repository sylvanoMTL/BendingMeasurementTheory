\subsection{Theory 1: Clamped Boundary Condition at Oarlock}
% ============================================
\label{sec:theory1}

\subsubsection{Boundary Conditions}

Theory 1 assumes the following boundary conditions:

\begin{itemize}
    \item \textbf{At the oarlock} ($x = 0$): The oar shaft is clamped, preventing both displacement and rotation:
    \begin{align}
    w_s(0) &= 0 \quad \text{(no vertical displacement)} \\
    \theta_s(0) &= 0 \quad \text{(no rotation, clamped)}
    \end{align}
    
    \item \textbf{At the handle} ($x = x_F$): A vertical force $F$ is applied, with no moment:
    \begin{align}
    V_s(x_F) &= -F \quad \text{(applied shear force)} \\
    M_s(x_F) &= 0 \quad \text{(free end, no moment)}
    \end{align}
\end{itemize}

See Figure~\ref{fig:theory1_boundary_conditions} for a visual representation of the boundary conditions.

\subsubsection{Deflection Solution}

\textbf{[TBD: Derivation of deflection $w_s(x)$ for clamped-free beam with point load]}

\subsubsection{Moment Distribution}

\textbf{[TBD: Calculation of bending moment $M_s(x)$ along the shaft]}

\subsubsection{Strain at Beam Location}

\textbf{[TBD: Calculation of mechanical strain $\varepsilon_{\text{mech}}$ at the beam gauge locations]}

\subsubsection{Numerical Example}

\textbf{[TBD: Numerical example with typical force values]}

% ============================================
